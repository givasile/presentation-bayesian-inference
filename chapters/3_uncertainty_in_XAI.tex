\begin{frame}
  \frametitle{Quantifying the uncertainty of the explanations}
  \begin{itemize}
    \item Many iml methods such as permutation feature importance or
    shaple values provide explanations without quantifying the
    uncertainty of the explanation
    \item The model itself, but also its
    explanations, are computed from data and hence are subject to
    uncertainty
  \item Interpretable Machine Learning – A Brief History,
    State-of-the-Art and Challenges by~\cite{molnar2020interpretable}
  spots it as a major open challenge in XAI domain
\end{itemize}
  \noindent\makebox[\linewidth]{\rule{\paperwidth}{0.4pt}}
  Uncertainty quantification is a major challenge in IML.
\end{frame}

\begin{frame}
  \frametitle{Could Bayesian Reasoning help?}
  \begin{itemize}
  \item We have to (re)define more rigorously the explainablity
    techniques
  \item Remember the two ingredients of Bayesian Reasoning; (a) Modeling
    and (b) Inference
  \item For (a), can we model explainability techniques as
    Probabilistic Graphical Models? (especially local surrogate models)
  \item If achieve (a), we can (re)search for (b) inference techniques
    that will provide explanations with uncertainty estimation
  \end{itemize}
  \noindent\makebox[\linewidth]{\rule{\paperwidth}{0.4pt}}
  Statement we can discuss: Bayesian Reasoning can be a recipe for
  uncertainty estimation in XAI

\end{frame}


\begin{frame}
  \frametitle{Feature effect}
  \begin{itemize}
  \item Some notation
    \begin{itemize}
    \item $x_s$ feature of interest
    \item $\mathbf{x_c}$ the rest of the features
    \item $i$, index pointing to the $i-th$ training example
    \end{itemize}

  \item ALE plots formula:
    \begin{gather}
      y = f_{ALE}(x) = \int_{x_0}^{x} \mathbf{E_{x_c|x_s}[f^{(1)}(x_s=z,x_c)]} \partial z\\
      f^{(1)}(x_s,x_c) =  \frac{\partial f}{\partial x_s}
    \end{gather}
    \item Intuition: \( y = f_{ALE}(x) = \int_{x_0}^{x} (\text{local effect at } z) \partial z \)
    \end{itemize}
  \noindent\makebox[\linewidth]{\rule{\paperwidth}{0.4pt}}
  ALE plots model the feature effect as the \alert{integration of the local feature effects}.

\end{frame}

\begin{frame}
  \frametitle{ALE approximation}
  \begin{itemize}
  \item ALE approximation:
    \begin{gather} \label{eq:ALE_appr}
      \hat{f}_{ALE}(x_s) = \sum_{k=1}^{K_{x_s}} \frac{1}{|N(k)|} \sum_{i:x_s^i \in N(k)} [f(z_k, \mathbf{x_c}^i) - f(z_{k-1}, \mathbf{x_c}^i)]\\
      \text{where} \; N(k) \text{ all the samples that belong to the k-th bin}
    \end{gather}

  \item ALE builds equally-sized bins and approximates the derivative
    by evaluating $f$ at the limits of the bins
  \item Side comment: when bins become large the approximation is bad,
    we work on a better one (stay up to date \dSmiley )
\end{itemize}
  \noindent\makebox[\linewidth]{\rule{\paperwidth}{0.4pt}}
  ALE approximation is weak when bins grow larger.
\end{frame}

\begin{frame}
  \frametitle{Quantify the variance}
  \begin{itemize}
  \item ALE plots formula:
    \begin{gather}
      y = f_{ALE}(x) = \int_{x_0}^{x} \mathbf{E_{x_c|x_s}[f^{(1)}(x_s=z,x_c)]} \partial z\\
      f^{(1)}(x_s,x_c) =  \frac{\partial f}{\partial x_s}
    \end{gather}
  \item ALE plots uses only with the expected local effect $\mathbf{E_{x_c|x_s}[f^{(1)}(x_s=z,x_c)]}$
  \item Intuition: \( y = f_{ALE}(x) = \int_{x_0}^{x} (\text{expected local effect at } z) \partial z \)
  \end{itemize}
  \noindent\makebox[\linewidth]{\rule{\paperwidth}{0.4pt}}
  ALE definition uses only the \alert{expected local effect} at each position
\end{frame}

\begin{frame}
  \frametitle{Quantify the variance}
  \begin{itemize}
  \item In general the local effect is a probability distribution \(p_{eff}(y|x_s)\)
  \item \(p_{eff}(y|x_s) = \int_{x_c} p(y,x_c|x_s) \partial x_c = \int_{x_c} p(y|x_c, x_s) p(x_c|x_s) \partial x_c\)
  \item \( p(y|x_c, x_s) = \delta (y - \frac{\partial f}{\partial x_s}(x_s,x_c)) \)
  \item \( p(x_c|x_s) \) is not known, but there are ways to estimate it?
  \item If estimating the whole distribution \(p_{eff}(y|x_s)\) is difficult, could we at least estimate some critical statistics
    \begin{itemize}
    \item expected value \( \mathbb{E}_{p_{eff}}[y] = \mu \)
    \item Variance \( \mathbb{E}_{p_{eff}}[(y - \mu)^2] \rightarrow \) uncertainty quantification
    \end{itemize}
  \item Details in the paper under construction \dSmiley
  \end{itemize}
  \noindent\makebox[\linewidth]{\rule{\paperwidth}{0.4pt}}
  If model the whole distribution of the local effect, we could estimate the uncertainty of the local effect.
\end{frame}

\begin{frame}
  \frametitle{Some first approaches in the domain}
  \begin{itemize}
  \item Feature importance
    \begin{itemize}
      \item Testing Conditional Independence in Supervised Learning Algorithms\cite{watson2021testing}
      \item All Models are Wrong, but Many are Useful\cite{JMLR:v20:18-760}
      \end{itemize}
    \item Relevance propagation
      \begin{itemize}
      \item On Feature Relevance Uncertainty: A Monte Carlo Dropout Sampling Approach~\cite{fabi2020}
      \end{itemize}
    \item Shapley value
      \begin{itemize}
      \item Efficient nonparametric statistical inference on
        population feature importance using Shapley
        values\cite{williamson2020efficient}
      \end{itemize}
    \end{itemize}
  \noindent\makebox[\linewidth]{\rule{\paperwidth}{0.4pt}}
  The domain of Uncertainty Quantification in XAI is still in its first steps
\end{frame}

\begin{frame}
  Thanks a lot for the attention!

  Let's discuss!
\end{frame}

\begin{frame}[allowframebreaks]
  \frametitle{Bibliography}
  \printbibliography
\end{frame}
